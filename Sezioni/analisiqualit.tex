In questa fase si effettua un'analisi delle qualità dello schema appena definito. Si suddividono in quattro categorie:
\begin{itemize}
\item \textbf{Correttezza:} Le entità sono usate in maniera consona e congruente per il loro scopo e non risultano errori sintattici e semantici.
\item \textbf{Completezza:} Il flusso di informazioni relativo ai processi interni dell'azienda è ben rappresentato dal modello. Questo significa che lo schema è completo e rappresenta i processi interni in maniera fedele rendendo naturale il percorso logico rappresentato.
\item \textbf{Leggibilità:} Lo schema presenta una buona complessità concentrata maggiormente nella regione centrale. Nonostante ciò il flusso di informazioni risulta spontaneo quindi la comprensibilità dello schema non è compromessa anche se sono stati necessari alcuni incroci per ottimizzare la disposizione di tutti i componenenti.
 \item \textbf{Minimalità:} Ogni specifica è utilizzata una sola volta all'interno dello schema quindi esso non contiene ridondanze.
\end{itemize}
