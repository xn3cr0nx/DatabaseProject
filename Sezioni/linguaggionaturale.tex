
Dopo aver analizzato le interviste effettuate e i documenti allegati, è stato possibile stabilire gli obiettivi che effettivamente vorremmo che la nostra base di dati raggiunga.
Il nostro scopo è realizzare un database che organizzi i dati di una azienda che vende prodotti e servizi sia a \hl{privati}, che a \hl{pubbliche amministrazioni} partecipando a gare pubbliche.

% Pensiamo che la nostra base dati duri almeno X anni, perchè bla bla ..

\noindent
\newline
Quindi si dovranno gestire i dati relativi a gare pubbliche del mercato elettronico, ai clienti, ai fornitori esterni, ai prodotti, ai servizi offerti, alle stipule di contratto.

\noindent
\newline
Siamo interessati a tenere traccia di tutte le \hl{gare pubbliche} a cui si partecipa (sia vinte, sia perse), per verificare la "forza economica" dell'azienda sul mercato e per conoscere i prezzi di vendita dei concorrenti.
Inoltre vogliamo tenere traccia dei prodotti e dei servizi più venduti dall'azienda stessa, per scopi statistici.

\noindent
\newline
Per quanto riguarda le gare pubbliche, si vogliono inserire dati relativamente ai prodotti/servizi richiesti, al tetto massimo di spesa, all'\hl{aggiudicatario della gara}, e al prezzo proposto dall'aggiudicatario della gara.

\noindent
\newline
Relativamente ai \hl{clienti}, si registreranno dati relativi alla ragione sociale e recapiti, e nel caso il cliente sia una pubblica amministrazione verrà inserito anche il codice PA e l'indirizzo PEC associati ad essa.

\noindent
\newline
Riguardo ai \hl{fornitori} dell'azienda, si vogliono inserire dati relativi alla ragione sociale, i recapiti, il \hl{catalogo prodotti} e la tabella per il calcolo delle \hl{spese di spedizione}.
Di ogni \hl{prodotto} vogliamo conoscere il codice prodotto, le caratteristiche generali (per poter partecipare a gare in cui non sono richiesti prodotti specifici), il prezzo (relativo a un dato fornitore), il peso e le dimensioni per il calcolo dei costi di spedizione.

\noindent
\newline
Relativamente ai \hl{servizi}, si vuole tenere traccia della lista dei servizi offerti, e la stima del costo di una prestazione (ad esempio: formattazione di 1 PC: 50 \euro, sostituzione di 1 hard disk: 35 \euro)

\noindent
\newline
Per ognuno dei \hl{contratti} stipulati, si vuole conoscere la controparte con cui questi vengono stipulati, la data, la tipologia (vendita prodotto, assistenza), l'importo, e, nel caso di contratti di assistenza on center, la data di inizio e scadenza del contratto.

\newpage
