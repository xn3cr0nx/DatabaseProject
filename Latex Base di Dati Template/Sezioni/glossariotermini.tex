%Se non si vuole creare la tablella in latex è possibile farla con word, copincollarla
%nella sezione File/paste table data del sito http://www.tablesgenerator.com
%che genererà la tabella in latex
%non scordare di mettere le chiusure di table e di adjustbox

% Comando per fare le celle su più righe, separa ogni riga con '\\'
\newcommand{\specialcell}[2]{
  \begin{tabular}[c]{@{}l@{}}#1\end{tabular}}

\begin{table}[h!]
  \centering
  \begin{adjustbox}{width=\textwidth}
    \begin{tabular}{@{}llll@{}}
      \toprule
      \textbf{TERMINE} & \textbf{DESCRIZIONE} & \textbf{SINONIMI} & \textbf{COLLEGAMENTO} \\ \midrule

      \multicolumn{1}{|l|}{Utente}
      & \multicolumn{1}{l|}{\specialcell{Persona fisica che gestisce e gestisce \\ e controlla il personaggio.}}
      & \multicolumn{1}{l|}{Giocatore}
      & \multicolumn{1}{l|}{Personaggio} \\ \midrule

      \multicolumn{1}{|l|}{Utente}
      & \multicolumn{1}{l|}{Mufasa dio}
      & \multicolumn{1}{l|}{Giocatore}
      & \multicolumn{1}{l|}{Personaggio} \\ \midrule

      \multicolumn{1}{|l|}{Utente}
      & \multicolumn{1}{l|}{\specialcell{Persona fisica che gestisce e gestisce \\ e controlla il personaggio.}}
      & \multicolumn{1}{l|}{Giocatore}
      & \multicolumn{1}{l|}{Personaggio} \\ \midrule

      \multicolumn{1}{|l|}{Utente}
      & \multicolumn{1}{l|}{\specialcell{Persona fisica che gestisce e gestisce \\ e controlla il personaggio.}}
      & \multicolumn{1}{l|}{Giocatore}
      & \multicolumn{1}{l|}{Personaggio} \\ \midrule


    \end{tabular}
  \end{adjustbox}
\end{table}
