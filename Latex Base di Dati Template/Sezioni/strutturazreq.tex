
\subsubsection{Frasi di Carattere Generale}
L'azienda necessità di interfacciarsi con diverse entità, fungendo da intermediaria indipendente, rispondendo alle richieste di gare pubbliche attraverso la disponibilità di fornitori privati. In tutto ciò la nostra base di dati può risultare fondamentale per l'ottimizzazione dei processi e in più per fornire importanti analisi su ciò che offre il mercato e quali migliorie possono essere apportate, così da ottimizzare non solo i processi, ma il business stesso.\newline
Si gestiranno perciò i dati riguardanti le gare pubbliche, i clienti, i fornitori, i prodotti, i servizi e tutte le tipologie di contratti stipulati tra l'azienda e i clienti.

\subsubsection{Frasi relative alle Gare Pubbliche}
Le gare pubbliche} vengono pubblicate sul sito acquistinretepa.it, cioè il Portale degli acquisti della Pubblica Amministrazione di proprietà del Ministero dell'Economia e delle Finanze e del Consip.\newline
La sezione di interesse per l'azienda è quella del Mercato Elettronico delle Pubbliche Amministrazioni, chiamato MEPA. Qui le imprese possono vedere le gare pubbliche e parteciparvi.\newline
Un'azienda abilitata alla vendita sul MEPA, necessita di credenziali di accesso al sito suddetto, in questo modo può partecipare ad un numero illimitato di gare.\newline
L'iscrizione a una gara consiste nel effettuare un'offerta, e dalla richiesta di offerta possono essere ottenute tutte le informazioni sulla richiesta e il richiedente, cioè la pubblica amministrazione che ha eseguito la richiesta, con tanto di codice identificativo PA.\newline
Quindi è importante registrare tutte le informazioni. L'idea è quella di registrare non solamente le gare aggiudicate, ma anche quelle perse, in modo da poter effettuare delle analisi successive. La quantità di gare presenti permette di effettuare questo tipo di operazione manualmente. Potrebbe essere previsto uno scraper in fase avanzata che gestisca in automatico questa procedura.\newline
La registrazione è determinata da due fasi, quella iniziale in cui si accingono i dati generali sulla gara e sul cliente, e poi una fase finale, dopo la chiusura della gara, per raccogliere dati sugli aggiudicatari e l'offerta effettuata, sempre a scopo di analisi.

\subsubsection{Frasi relative ai Clienti}
Per quel che riguarda i clienti, si hanno sia enti privati che pubbliche amministrazioni. Per gli enti privati sono necessari i dati relativi alla ragione sociale, l'indirizzo, dati di fatturazione. Per le pubbliche amministrazioni, oltre a questi dati, servono informazioni riguardo al codice PA e all'indirizzo di posta PEC, tutte informazioni estraibili dalle richieste di offerta sul MEPA. \newline
Sia ai privati e sia alle pubbliche amministrazioni saranno poi legati i contratti e le fatture, nelle apposite tabelle.

\subsubsection{Frasi relative alle Assistenze}
Le assistenze si dividono in assistenze di due tipi:
\begin{itemize}
\item	assistenze on center;
\item	assistenze a chiamata;
\end{itemize}
Le assistenze on center sono erogate a domicilio, in seguito alla stipulazione di un contratto di durata prestabilita, solitamente dai quattro mesi a un anno, per cui il cliente paga una quota fissa in modo da ricevere assistenza sul luogo in base alle sue esigenze.\newline
I dati importanti al riguardo sono il costo del contratto, la data di inizio e la data di scadenza, e il tipo di assistenza.\newline
Le assistenze a chiamata non sono vincolate da uno contratto, il cliente può chiedere interventi su richiesta e paga a prestazione compiuta.
In questo caso si registrano la data in cui si è fatta assistenza, il tipo di servizio erogato, e il cliente che ha ricevuto assistenza.


\subsubsection{Frasi relative ai Prodotti}
I prodotti sono quelli messi a disposizione dai fornitori, i quali forniscono i propri cataloghi che comprendono una lista di codici prodotto con il relativo prezzo di vendita. I dati quindi da registrare sono soprattutto il codice prodotto, in quanto spesso i clienti chiedono un prodotto in base al codice specifico, e le caratteristiche tecniche del prodotto.\newline
In seguito alla vendita di un prodotto viene emessa una fattura al cliente con indicati i prodotti venduti, i costi, e i dati relativi a venditore e acquirente.

\subsubsection{Frasi relative ai Servizi}
I servizi sono entità particolari che necessitano di una standardizzazione nella loro trattazione in quanto il costo è molto variabile e non è determinato per forza in base al tempo speso per effettuare tale servizio. I servizi devono essere registrati con il loro costo, e deve essere strutturata una descrizione generica da poter riportare. Il servizio di per sè implica anche uno spostamento di cui tener conto, che potrebbe essere comparato a quelli che sono i costi di spedizione in caso di acquisto di un prodotto.\newline
Dopo l'erogazione di un servizio viene emessa una fattura al cliente che deve riportare la prestazione eseguita, il costo, e i dati relativi al fornitore e al ricevente del servizio.

\subsubsection{Frasi relative ai Fornitori}
I fornitori sono coloro da cui vengono acquistati i prodotti da vendere ai clienti. I dati necessari al riguardo sono simili a quelli dei clienti, quindi bisogna registrare ragione sociale, indirizzo, dati di fatturazione per i pagamenti. Ad essi è importante legare i catalogi forniti, che vengono consultati per verificare la disponibilità dei prodotti, e vengono aggiornati mensilmente.

\subsubsection{Frasi relative alle Fatture}
Si vogliono conoscere le informazioni relative a una fattura, sia in entrata che in uscita. Sono quindi necessari il codice del contratto a cui si riferisce, la data di emissione, la data entro cui è necessario pagare.

% \subsubsection{Frasi relative agli Ordini}
% Gli ordini rappresentano i prodotti acquistati dai fornitori. Spesso gli ordini effettuati non sono spediti all'azienda che deve poi spedirli ai clienti, ma vengono direttamente spediti al cliente, seguendo il modello di dropshipping. In ogni caso bisogna tener traccia anche dei costidi spedizone}, oltre che al prodotto interessato dall'ordine e le sue caratteristiche, il costo dell'ordine, il fornitore del prodotto, il cliente che ha acquistato il prodotto per cui è stato effettuato l'ordine. Tutto ciò implica la connessione di diverse entità interessate che insieme vanno a formare l'ordine. Qui come nelle trattative bisogna considerare la fattura, in particolare la fattura emessa ai clienti da parte della azienda, e in questo caso la fattura emessa dal fornitore all'azienda per l'acquisto.
