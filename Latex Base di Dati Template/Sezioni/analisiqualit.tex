
Analizzeremo ora le qualità piu importanti dello schema concettuale:
\begin{itemize}
\item \textbf{Correttezza:} Non sono presenti errori sintattici o semantici, le entità e le relazioni risultano usate correttamente al loro scopo.
\item \textbf{Completezza:} Lo schema ha raggiunto un livello di completezza che giudichiamo soddisfacente, esso infatti attraversa e desccrive tutti i dati fondamentali per i processi interni al gioco, nella parte che riguarda l'interfaccia dell'utente con il sito non siamo entrati molto nel dettaglio in quanto le nostre specifiche erano maggiormente orientate al Gioco
\item \textbf{Leggibilità:}Nel nostro schema le Relazioni per la loro natura sono piuttosto intrecciate, abbiamo cercato di aumentare la Leggibilità introducendo delle variazioni di colore tra Entità genitori e figli, cercando un posizionamento favorevole delle stesse per generare meno intrecci possibile, tramite alcune prove con persone estranee al progetto abbiamo verificato la corretta comprensione dello schema.
 \item \textbf{Minimalità:}Lo Schema risulta privo di ridondanze, ogni specifica è rappresentata una sola volta al suo interno.
\end{itemize}
